\documentclass{article}
\usepackage[top=2cm,bottom=2cm,left=2cm,right=2cm]{geometry}
\usepackage{amsmath, amssymb, amsthm}
\usepackage{hyperref}
\usepackage{csquotes}

\usepackage{datetime}
\newdateformat{petsa}{\the\day\ \shortmonthname[\the\month] \the\year}

\newtheorem{definition}{Kahulugan}[section]
\newtheorem{theorem}{Teyorem}[section]
\newtheorem{lemma}[theorem]{Lema}
\newtheorem{problem}[theorem]{Problema}
\renewcommand{\abstractname}{Abstrak}
\renewcommand{\refname}{Sanggunian}

\title{Paghuhulog at Di Pagkamainam ng Batay}
\author{poypoyan}
\date{11 Nov 2024}   % \petsa\today

\begin{document}

\maketitle

\begin{abstract}
Tutal wala namang tumitingin ng repositoryong naglalaman ng payl na ito kundi ako, isinulat ko na lámang ang papel na ito sa aking sinusung wika (at sa pamamaraang trip ko). Marahi'y mainam rin itong pagsubok para sa akin sa pagsulat ng matematika sa Tagalog.

Maaari itong ituring na kasunod sa talâ kong ``My Pet Set Theory: \textsf{AST+NWF}" \cite{ast-nwf}. Inilatag dito ang paghuhulog (\textit{defining}) ng kahulugan sa Unang Antas na Lohika (\textit{First-order Logic}, iaakronim dito bilang UAL), at ng Di maInam ang Batay (\textit{Non-well-founded}, iaakronim dito bilang DIB) bilang isang anyo ng paghuhulog.
\end{abstract}

\section{Ang Kahulugan ng Kahulugan}

Hindi ko pa nakikita sa ibang panitikan ang pagpapahayag ko ng DIB sa \textsf{AST+NWF}, kayâ kailangan ng paglilinaw sa kung ano ang nangyayari sa teyorya kapagka naghuhulog ng ganoon. Pagtutuunan ko lámang ng pansin sa papel na ito ang UAL sapagkat 1) ito ang batayang lohika ng \textsf{AST+NWF}, at 2) ito ang pinakapinag-aaralang lohika sa larang ng lohikang matematika. Maraming pamamaraan upang maisapormal ang DIB; isang halimbawa ay ang mula sa grap. Subali't sa palagay ko, ang sumusunod ang pinaka nasa diwa ng lohikang matematika, sapagkat gumagamit ng ``pangungusap".

Maglatag ng isang unang antas (UA) na teyorya $\textsf{T} = (\sigma, \mathcal{L}_\sigma, \mathcal{R})$ nks\footnote{na kung saan} ang $\sigma$, $\mathcal{L}_\sigma$ at $\mathcal{R}$ ay ang tipón (\textit{set}) ng mga sariling tanda (\textit{signatures}), wikang UA na may mga tandang iyon, at mga aksyom at tuntuning panghulo (\textit{inference rules}), ayon sa pagkakabanggit.

\begin{definition}[Teyoryang may bagong panaguri]
\label{paghuhulog}
Ang $\textsf{T+p} = (\sigma \cup \{\textsf{p}\}, \mathcal{L}_{\sigma \cup \{\textsf{p}\}}, \mathcal{R} \cup \{\alpha\})$ ay isang teyoryang gawa sa \textsf{T} na may bagong panaguri (\textit{predicate}) \textsf{p} nks ang $\alpha$ ay ang sumusunod na aksyom ng paghuhulog ng panaguri
\begin{align}
\forall z_0, \ldots, z_{n-1}(\textsf{p}(z_0, \ldots, z_{n-1}) \leftrightarrow a(z_0, \ldots, z_{n-1})).
\end{align}
nks $a \in \mathcal{L}_\sigma$.
\end{definition}

Ang gayong panaguri ay akmang tawaging ``mainam ang batay" sapagkat maaaring ``ibalik" ang bagong teyorya túngo sa \textsf{T}---ang lahat ng \textsf{p} sa $\mathcal{L}_{\sigma \cup \{\textsf{p}\}}$ ay papalitan ng laman ng $a$, at saka tatanggalin ang mismong \textsf{p} at aksyom (1). Itong pagbabalik rin ang isa sa dahilan kung bakit di nagdadalawang-isip ang mga teyoretista ng modelo na pag-aralan ang mga teyoryang walang kahulugan lámang, kahit sinasabi ng \ref{paghuhulog} na sa bawat bagong kahulugan ay may bagong teyorya. Ang isa pang dahilan ay nasa huling seksyon.

Ngayon, paano na maghulog ng panaguri na DIB? May dalawang paraan:

\begin{enumerate}
\item Ang $a$ sa aksyom (1) ay mayroon nang \textsf{p} sa loob nito (kayâ $a \in \mathcal{L}_{\sigma \cup \{\textsf{p}\}}$).
\item Isang bagsakang maghulog ng walang hanggang (nabibilang (\textit{countable})) na mga panaguri $\textsf{p}_{0}, \textsf{p}_{1}, \textsf{p}_{2}, \ldots$, at sundin lámang ang \ref{paghuhulog}.
\end{enumerate}

\noindent Akma ang pagtawag ng ``di mainam ang batay" sapagkat \textit{hindi} na maaaring gawin ang kasusulat lámang na pagbabalik sa lumang teyorya.

Mula sa unang paraan nga ang anyo ng $\phi$ sa \textsf{NWFAck} ng \cite{ast-nwf}---hinulog ang $\phi$ nks nasa loob ang $\phi$ sa $a$. Tila bagá may ``salaming" tinitignan ang panaguri upang makita at mailarawan ang sarili kasama ng ibang nakikita sa salamin.

Mula sa ika-2ng paraan naman ang paghuhulog ng panaguri na di lang basta binabanggit ang sarili (\textit{self-referential}), para ng balintuna ni Yablo\cite{yablo} sa halip ng balintuna ng sinungaling. Datapwat káya pa rin ng \textsf{AST+NWF} na iwari ang ika-2ng paraan sa pamamagitan ng unang paraan: kailangan ng $\phi$ ng isang pantanging baryabol $z_0\in\mathbb{N}$ (kumbaga $\phi(z_0, \ldots)$) nks ang bawat instansya ng $\phi$ sa loob ng $a$ ay may anyong $\phi(z'_0, \ldots), z'_0 \in\mathbb{N}$.

\section{Isang Pagtutumbas para sa mga Panaguri}

Ang unang paraan lang ang pagtutuunan dito. Ang sumusunod ay isang paghuhulog ng tumbasan sa pagitan ng mga panaguring DIB.

\begin{definition}
\label{tumbasan}
Ang dalawang panaguring DIB $\textsf{p}_1$ at $\textsf{p}_2$ nks iisa ang dami ng baryabol ay ituturing na \textbf{iisa} kung mula sa pag-aakala na iisa ang instansya ng $\textsf{p}_1$ at ng $\textsf{p}_2$ kkiln\footnote{kung at kung ito lang na (if and only if)} iisa ang anyo ng pinasok sa dalawa (halimbawa, $\textsf{p}_1(a, b, c) \leftrightarrow \textsf{p}_2(a, b, c)$ at $\neg(\textsf{p}_1(a, b, c) \leftrightarrow \textsf{p}_2(a, c, b))$), ay mapatutunayang $a_1 \leftrightarrow a_2$.
\end{definition}

Ayto ang isang kongkretong halimbawa: ang $$\textsf{p}_1(z_0, z_1) \leftrightarrow \neg\forall x(\textsf{p}_1(x, z_0) \rightarrow \textsf{p}_1(x, z_1))$$ at $$\textsf{p}_2(z_0, z_1) \leftrightarrow \exists y(\neg \textsf{p}_2(y, z_1) \wedge \textsf{p}_2(y, z_0))$$ ay iisa. Ang inakalang tumbasan ay $\textsf{p}_1(x, z_0) \leftrightarrow \textsf{p}_2(y, z_0)$ at $\textsf{p}_1(x, z_1) \leftrightarrow \textsf{p}_2(y, z_1)$.

Ibig sabihin, kung $\textsf{p}_1 \leftrightarrow \textsf{p}_2$, di $\textsf{T+}\textsf{p}_1\textsf{+p}_2 = \textsf{T+}\textsf{p}_1$! Sa kaso ng \textsf{AST+NWF}, ang pagtatakda nitong katumbasan ay nagdudulot ng pagkanatatangi ng mga DIB na tipón. Dahil dito'y ibig ko ngang itakda ito kapagka ginagamit ang teyorya.  

\section{Sa Ibáyo ng Karaniwang Paghuhulog ng Istruktura?}

Sa palagay ko, kayâ napakadalang pag-aralan ng di pagkamainam ng batay ay dahil sa mainam ang batay ng mga istrukturang pinag-aaralan ngayon sa matematika. Halimbawa, ang \textit{pangkat} $\mathbb{G} = (G, \cdot, 1)$ ay binubuo ng isang tipón $G$, dalawahang operasyon/punsyon $\cdot:G \times G \rightarrow G$, at isang $1 \in G$ na maaaring tawaging ``di nang-iiba" o ``neyutro". Ang pagkamainam ng batay ay makikita sa waring dapat na ``naunang nahulog" ang tatlong bumubuo kaysa $\mathbb{G}$. Sa ibang sabi, hindi kailanman ginawang tumbas sa $\mathbb{G}$ ang alinman sa tatlo!

Ang pag-aalis ng paghihigpit na ito ang maaaring unang hakbang upang makapaghulog ng mga matematikang istruktura na DIB. Sa totoo lang, kahit ako'y hindi gaanong naniniwalang may ihuhulog sa hinaharap na mahahalagang istrukturang ganito. Sa huli, sa tingin ko, ito'y isang ``dulo" na ng pilosopiya ng teyorya ng tipón na ``ang bagay ay binubuo ng iba pang bagay". 

\bibliographystyle{plain}
\bibliography{paghuhulog}
\end{document}
