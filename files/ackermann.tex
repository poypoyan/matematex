\documentclass{article}
\usepackage[top=2cm,bottom=2cm,left=2cm,right=2cm]{geometry}
\usepackage{amsmath, amssymb, amsthm}
\usepackage{hyperref}
\usepackage{csquotes}

\usepackage{datetime}
\newdateformat{petsa}{\the\day\ \shortmonthname[\the\month] \the\year}

\newtheorem{definition}{Definition}[section]
\newtheorem{theorem}{Theorem}[section]
\newtheorem{lemma}[theorem]{Lemma}
\newtheorem{problem}[theorem]{Problem}

\title{My Pet Set Theory: \textsf{AST+NWF}}
\author{poypoyan}
\date{27 Dec 2023}   % \petsa\today

\begin{document}

\maketitle

\begin{abstract}
This note presents a variant of Ackermann Set Theory with easy construction of non-well-founded (NWF) sets. Motivations for using up our time for this are also laid out.
\end{abstract}

\section{The Theory}
We will now describe a variant of Ackermann Set Theory \cite{ackermann} we denote as \textsf{AST+NWF} (let's denote the original theory as \textsf{oAST}). \textsf{AST+NWF} is formulated in first-order logic with equality and with a constant $V$ which is interpreted as the set universe, and a binary relation $\in$ which is interpreted as the usual membership relation.

\begin{definition}[Super-Completeness of $V$]
Let $SC(V)$ be the statement $$\forall x, y (x \in V \wedge (y \in x \vee y \subseteq x) \rightarrow y \in V)$$ where $\subseteq$ is the usual subset relation, defined as $x \subseteq y \leftrightarrow \forall z(z \in x \rightarrow z \in y).$
\end{definition}

\noindent Now here are the axioms of $\textsf{AST+NWF}(V)$:

\begin{enumerate}
    \item Axiom of Extensionality $\textsf{Ext}(V)$: $$\forall x, y (x \in V \wedge y \in V \wedge \forall z(z\in x \leftrightarrow z \in y) \rightarrow x = y).$$
    \item Ackermann Schema $\textsf{Ack}(V)$: Let $\phi(y, z_0, \ldots, z_{n-1})$ be a unary first-order formula where all the constants $z_0, \ldots, z_{n-1} \in V$. Then $$(SC(V) \rightarrow \forall y(\phi(y, \ldots) \rightarrow y \in V)) \rightarrow \exists x(x \in V \wedge \forall y(y \in x \leftrightarrow \phi(y, \ldots))).$$
    \item \textbf{Non-well-founded} Ackermann Schema $\textsf{NWFAck}(V)$: Let $\phi(y, z_0, \ldots, z_{n-1})$ be a unary first-order formula where all the constants $z_0, \ldots, z_{n-1} \in V$, and it is of the form
$$\forall x_0, \ldots ((\forall y_0(y_0 \in x_0 \leftrightarrow \phi(y_0, \ldots_0)) \wedge \ldots) \rightarrow \ldots)$$
where all those instances of $\phi$\footnote{Yes, this is a recursive definition!} are also \textit{defined}, and in the last ellipsis, the $x$'s only occur in equality relation (e.g., $y = x_0$). Then if
$$(\bigwedge_i \forall y_i( \phi(y_i, \ldots_i) \rightarrow y_i \in V)) \rightarrow \\ (SC(V) \rightarrow \forall y(\phi(y, \ldots) \rightarrow y \in V)),$$
then
\begin{align*}
    \exists x(x \in V \wedge \forall y(&y \in x \leftrightarrow \phi(y, \ldots))) \\ \wedge\ &\forall y, c_0, \ldots, c_{n-1} ((\phi(y, z_0, \ldots, z_{n-1}) \leftrightarrow \phi(y, c_0, \ldots, c_{n-1})) \rightarrow \bigwedge_{j=0}^{n-1} {c_j = z_j}).
\end{align*}
\end{enumerate}

\section{Motivations}
There are several motivations for the ``design" of the theory.\\

The first motivation is the focus on \textit{set constructions} instead of sets and/or proper classes. Because of this, we decided to:
\begin{itemize}
\item Adopt \textsf{oAST}. The Ackermann schema captures the idea that ``natural"/``uncontroversial" set constructions are 1) definable as a first-order sentence (hence, are ``finite''), and 2) ``universe-agnostic'' (since those do not mention $V$). This reminds us while studying infinities that our ``full'' descriptions of objects are always finite.

On the contrary, consider the Axiom of Choice (AC), a well-known axiom independent to \textsf{ZF}. The sets it constructs are not unique, hence it is said to postulate existence of sets without defining it, unlike other \textsf{ZF} (set construction) axioms \cite[Chapter 5]{jech}. However, if we just permit ``lengths'' of any ordinal to $\phi$ in Ackermann schema, AC can now ``produce'' unique sets again! To see this, let $S$ be a set of sets, then an appropriate $\phi$ for AC would simply be $$y = a_0 \vee y = a_1 \vee y = a_2 \vee \ldots$$ where $a_0 \in s_0, a_1 \in s_1, a_2 \in s_2, \ldots$ and $S=\{s_0, s_1, s_2, \ldots\}$. Nevertheless, stronger axioms like Choice can be added to our theory via relativization to $V$ wherein every instance of $\forall x \varphi$ in the statement is replaced by $\forall x (x \in V \rightarrow \varphi)$ and every instance of $\exists x \varphi$ is replaced by $\exists x (x \in V \wedge \varphi)$.
\item Remove the Class Construction Schema for our \textsf{AST+NWF}. Note that Class Construction Schema \textit{is} Separation Schema for $V$ and that Separation Schema immediately follows from Ackermann schema by setting $\phi$ to $y \in a \wedge \varphi$ for $a\in V$.
\end{itemize}

The next motivation is to make the universe(s) ``as closed as possible". Because of this, we decided to:
\begin{itemize}
\item Restrict Extensionality to sets ($x \in V$) only.
\item Put super-completeness inside Ackermann schemas instead of it being an axiom on its own. This is done so that when we work on multiple universes $V_0, V_1, \ldots$ such that $V_0 \in V_1 \in \ldots,$ and $\textsf{AST+NWF}(V_0) \wedge \textsf{AST+NWF}(V_1) \wedge\ldots,$ no additional sets in $V_0$ will be shown to exist through the higher universes (i.e., $V_1, \ldots$).
\end{itemize}

Now the NWF Ackermann schema is formulated for easy contruction of NWF sets (obviously), but it is a beast! Nevertheless, the intuition is actually simple:

\begin{displayquote}
If instances of $\phi$ are used to define itself, and we assume that all that satisfy those instances of $\phi$ are all sets, then $\phi$ can indeed construct a set, and the constructed set is \textit{unique through the given parameters}.
\end{displayquote}

For example, to construct a set such that $x=\{x, a\}$ where $a \in V$, $\phi(y, a)$ is defined as $$\forall x_0 (\forall y_0 (y_0 \in x_0 \leftrightarrow \phi(y_0, a)) \rightarrow y = x_0 \vee y = a).$$ Notice that this satisfies the prescribed form for $\phi$. This is motivated by what we'll call the \textit{black box principle}: elements and subsets of an NWF set should \textit{never} be used to build that set, for two reasons: 1) to prevent constructing properties ``so loose" that well-founded sets can satisfy it (i.e., $\omega$ is a set such that $\bigcup x = x$), and 2) to prevent inconsistency (i.e., a set such that $2^x = x$ contradicts Cantor's theorem). Next, the statement ``all those instances of $\phi$ are also \textit{defined}" means that $x(\emptyset) = \{x(\{\emptyset\}), x(\emptyset)\}$ for example, will not be defined unless $x(\{\emptyset\})$ is defined first.

The last highlight in the above quote, the statement ``unique through the given parameters", needs explanation. Our main motivation for this is the ``intuitive" set-theoretic definition of ordered pair $(x,y) = \{(0,x), (1, y)\}$. For this to satisfy the ordered pair property $(x_0,y_0) = (x_1, y_1) \leftrightarrow x_0 = x_1 \wedge y_0 = y_1$ in our theory, the sets $(0,x)$ and $(1,y)$ should \textit{always} be unequal. Since by the Opacity Principle we cannot ``view inside" $\phi$, there is freedom on equalities of elements across NWF sets. Hence the statement can be seen as either a ``cheat", or the most natural generalization of the always inequality between $(0,x)$ and $(1,y)$.

Lastly, the nature of recursive defining of $\phi$ in the schema needs further consideration. Maybe \textsf{AST+NWF} is not a first-order theory anymore because recursive definition of predicates causes its signature to have ``incrementing" quantity of predicates.

\section{What About Replacement?}
All set construction axioms of \textsf{ZF} (relativized to $V$) are straightforward to prove using Ackermann schema, except the Axiom Schema of Replacement. When Reinhardt in \cite{reinhardt} proved that \textsf{ZF} can be derived from \textsf{oAST} plus Regularity, the whole proof (which we don't understand) really boils down to proving Replacement! Regularity is used in the proof, hence that work only applies to well-founded sets.

Here's the Replacement Schema, relativized to $V$:
$$\forall w(w \in a \rightarrow \exists! y(y \in V \wedge \varphi(w, y, \ldots))) \rightarrow \exists x(x \in V \wedge \forall y(y \in x \leftrightarrow y \in V \wedge \exists w(w \in a \wedge \varphi(w, y, \ldots))))$$
where $a \in V$ and $\varphi$ is relativized to $V$, but all its other constants are in $V$. What about our variant? \textit{We still don't know for sure}\footnote{This note became the most edited document in the repository because I change my answer to that question everytime it comes to my mind, from ``definitely no" to ``I don't know", and back and forth.}. The direction for potential proof is to come up with a general transformation of $\varphi$ to a new predicate $\phi$ such that its constants are now all in $V$\footnote{It needs to be established that $V\notin V$. This can be readily proved by constructing the Russell set using Ackermann schema.} and $\phi \leftrightarrow y \in V \wedge \exists w( w \in a \wedge \varphi(w, y, \ldots))$. All we know is that there are cases wherein ``simply removing $V$" works. That is, the $\phi$ will be $\exists w( w \in a \wedge \varphi'(w, y, \ldots))$ where $\varphi'$ is the de-relativized $\varphi$.

Nonetheless, we have a weaker variant of Replacement at home:
\begin{align*}
    \forall w(w \in a \rightarrow \exists y(y \in V \wedge \forall v &(v \in y \leftrightarrow \varphi(v, w, \ldots)))) \\ \rightarrow \exists x &(x \in V \wedge \forall y(y \in x \leftrightarrow \exists w(w \in a \wedge \forall v (v \in y \leftrightarrow \varphi(v, w, \ldots)))))
\end{align*}
where $a \in V$ and the constants of $\varphi$ are all in $V$. The relation between this schema and the original Replacement is akin to the relation between the first-order induction schema and the second-order induction axiom in arithmetic. In the antecedent, the weaker variant demands ``explicit" existence and uniqueness of sets, whereas the original only demands \textit{the} existence and uniqueness of sets, regardless of how those came about. We expect that this is the useful variant when working in our theory, or in other words, the Replacement ``in practice".

\bibliographystyle{plain}
\bibliography{ackermann}
\end{document}
