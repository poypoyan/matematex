\documentclass{article}
\usepackage[top=2cm,bottom=2cm,left=2cm,right=2cm]{geometry}
\usepackage{amsmath, amssymb, amsthm}
\usepackage{hyperref}
\usepackage{csquotes}

\usepackage{datetime}
\newdateformat{petsa}{\the\day\ \shortmonthname[\the\month] \the\year}

\newtheorem{definition}{Definition}[section]
\newtheorem{theorem}{Theorem}[section]
\newtheorem{lemma}[theorem]{Lemma}
\newtheorem{problem}[theorem]{Problem}

\title{My Pet Set Theory: \textsf{AST+NWF}}
\author{poypoyan}
\date{\petsa\today}

\begin{document}

\maketitle

\begin{abstract}
In this note we present a variant of Ackermann Set Theory (AST) with easy construction of non-well-founded sets (NWF). Motivations for using up our time for this are also laid out.
\end{abstract}

\section{The Theory}
We will now describe a variant of Ackermann Set Theory \cite{ackermann} we denote as \textsf{AST+NWF}. \textsf{AST+NWF} is formulated in first-order logic with equality and with a constant $V$ which is interpreted as the set universe, and a binary relation $\in$ which is interpreted as the usual membership relation.

\begin{definition}[Super-Completeness of $V$]
Let $SC(V)$ be the statement $$\forall x, y (x \in V \wedge (y \in x \vee y \subseteq x) \rightarrow y \in V)$$ where $\subseteq$ is the usual subset relation, defined as $x \subseteq y \leftrightarrow \forall z(z \in x \rightarrow z \in y).$
\end{definition}

\noindent Now here are the axioms of $\textsf{AST+NWF}(V)$:

\begin{enumerate}
    \item Axiom of Extensionality $\textsf{Ext}(V)$: $$\forall x, y (x \in V \wedge y \in V \wedge \forall z(z\in x \leftrightarrow z \in y) \rightarrow x = y).$$
    \item Ackermann Schema $\textsf{Ack}(V)$: Let $\phi(y, z_0, \ldots, z_{n-1})$ be a unary first-order formula where all the constants $z_0, \ldots, z_{n-1} \in V$ and $\ne V$. Then $$(SC(V) \rightarrow \forall y(\phi(y, \ldots) \rightarrow y \in V)) \rightarrow \exists x(x \in V \wedge \forall y(y \in x \leftrightarrow \phi(y, \ldots))).$$
    \item \textbf{Non-well-founded} Ackermann Schema $\textsf{NWFAck}(V)$: Let $\phi(y, z_0, \ldots, z_{n-1})$ be a unary first-order formula where all the constants $z_0, \ldots, z_{n-1} \in V$ and $\ne V$, and it is of the form\footnote{Yes, this is a recursive definition!}
$$\forall x_0, \ldots ((\forall y_0(y_0 \in x_0 \leftrightarrow \phi(y_0, \ldots_0)) \wedge \ldots) \rightarrow ([y = x_i \vee] \ldots))$$
where in the last ellipsis, sub-formulas of form $\forall a \ldots (\forall b (b \in a \leftrightarrow ([b = x_i \vee] \ldots)))$ may occur, and all those instances of $\phi$ in $\phi(y, \ldots)$ are also \textit{defined}. Note that $[\cdot]$ means that what is inside it may or may not be there in the formula. Then if 
$$(\bigwedge_i \forall y_i( \phi(y_i, \ldots_i) \rightarrow y_i \in V)) \rightarrow \\ (SC(V) \rightarrow \forall y(\phi(y, \ldots) \rightarrow y \in V)),$$
then
\begin{align*}
    \exists x(x \in V \wedge \forall y(&y \in x \leftrightarrow \phi(y, \ldots))) \\ \wedge\ &\forall y, c_0, \ldots, c_{n-1} ((\phi(y, z_0, \ldots, z_{n-1}) \leftrightarrow \phi(y, c_0, \ldots, c_{n-1})) \rightarrow \bigwedge_{j=0}^{n-1} {c_j = z_j}).
\end{align*}
\end{enumerate}

\section{Motivations}
There are several motivations for the ``design" of the theory.\\

The first motivation is the focus on \textit{set constructions} instead of sets and/or proper classes. Because of this, we decided to:
\begin{itemize}
\item Adopt AST in the first place. The Ackermann schema captures the idea that ``natural"/``uncontroversial" set constructions are 1) definable as a first-order sentence (hence, are ``finite''), and 2) ``universe-agnostic'' (since those do not mention $V$). This reminds us while studying infinities that our ``full'' descriptions of objects are always finite.

On the contrary, consider the Axiom of Choice (AC), a well-known axiom independent to ZF. The sets it constructs are not unique, hence it is said to postulate existence of sets without defining it, unlike other ZF (set construction) axioms \cite[Chapter 5]{jech}. However, if we just permit ``lengths'' of any ordinal to $\phi$ in Ackermann Schema, AC can now ``produce'' unique sets again! To see this, let $S$ be a set of sets, then an appropriate $\phi$ for AC would simply be $$y = a_0 \vee y = a_1 \vee y = a_2 \vee \ldots$$ where $a_0 \in s_0, a_1 \in s_1, a_2 \in s_2, \ldots$ and $S=\{s_0, s_1, s_2, \ldots\}$. Nevertheless, stronger axioms like Choice can be added to our theory via relativization to $V$ wherein every instance of $\forall x \varphi$ in the statement is replaced by $\forall x (x \in V \rightarrow \varphi)$ and every instance of $\exists x \varphi$ is replaced by $\exists x (x \in V \wedge \varphi)$.
\item Remove the Class Construction Schema for our \textsf{AST+NWF}. Note that Class Construction Schema \textit{is} Separation Schema for $V$ and that Separation Schema immediately follows from Ackermann Schema by setting $\phi$ to $y \in a \wedge \varphi$ for $a\in V$.
\end{itemize}

The next motivation is to make the universe(s) ``as closed as possible". Because of this, we decided to:
\begin{itemize}
\item Restrict Extensionality to sets ($x \in V$) only.
\item Put super-completeness inside Ackermann schemas instead of it being an axiom on its own. This is done so that when we work on multiple universes $V_0 \subset V_1 \subset \ldots$ where $ \textsf{AST-NWF}(V_0) \wedge \textsf{AST-NWF}(V_0) \wedge\ldots$, no additional sets in $V_0$ will be shown to exist through the higher universes $V_1, \ldots$, and \textit{not} through Ackermann schemas for $V_0$.
\end{itemize}

Now the Non-well-founded Ackermann Schema is formulated for easy contruction of non-well-founded sets (obviously), but it is a beast! Nevertheless, the intuition is actually simple:

\begin{displayquote}
If instances of $\phi$ are used to define itself, and we assume that all that satisfy those instances of $\phi$ are all sets, then $\phi$ can indeed construct a set, and the constructed set is \textit{unique through the given parameters}.
\end{displayquote}

For example, to construct a set such that $x=\{x, a\}$ where $a \in V$, $\phi(y, a)$ is defined as $$\forall x_0 (\forall y_0 (y_0 \in x_0 \leftrightarrow \phi(y_0, a)) \rightarrow y = x_0 \vee y = a).$$ Notice that this satisfies the prescribed form for $\phi$. This is motivated by what I'll call the \textit{Opacity Principle}: elements and subsets of a non-well-founded set should \textit{never} be used to build that set, for three reasons: 1) the elements (and subsets consequently) are not yet defined before $\phi$ is stated, 2) to prevent constructing properties that well-founded sets can satisfy (i.e., $\omega$ is a set such that $\bigcup x = x$), and 3) to prevent inconsistency (i.e., a set such that $2^x = x$ contradicts Cantor's theorem). Next, the statement in the axiom ``all those instances of $\phi$ in $\phi(y, \ldots)$ are also \textit{defined}" is a reference to ``mutually defined non-well-founded sets" like the two sets $x(0)$ and $x(1)$ such that $x(1) = \{x(0), a\}$ and $x(0) = \{x(1), b\}$ where $a, b \in V$. Without satisfying this statement, $x(1)$ will be defined using $x(0)$ which is undefined, and hence both sets are meaningless.

The last highlight in the above quote, the statement ``unique through the given parameters", needs explanation. Our main motivation for this is the ``intuitive" set-theoretic definition of ordered pair $(x,y) = \{(0,x), (1, y)\}$. For this to satisfy the ordered pair property $(x_0,y_0) = (x_1, y_1) \leftrightarrow x_0 = x_1 \wedge y_0 = y_1$ in our theory, the sets $(0,x)$ and $(1,y)$ should \textit{always} be unequal. Since by the Opacity Principle we cannot ``view inside" $\phi$, there is freedom on equalities of elements across non-well-founded sets. Hence the statement can be seen as either a ``cheat", or the most natural generalization of the always inequality between $(0,x)$ and $(1,y)$.

Lastly, the nature of recursive defining of $\phi$ in the schema needs further consideration. Maybe \textsf{AST+NWF} is not a first-order theory anymore because recursive definition of predicates causes its signature to have ``incrementing" quantity of predicates. I think that this shows that the concept of ``definition" itself can be tricky in formal logic, and hence its ``effects" on theories and logics also need further consideration.

\bibliographystyle{plain}
\bibliography{ackermann}
\end{document}
