\documentclass{article}
\usepackage[top=2cm,bottom=2cm,left=2cm,right=2cm]{geometry}
\usepackage{amsmath, amssymb, amsthm}
\usepackage{tikz}
\usepackage{hyperref}

\usepackage{datetime}
\newdateformat{petsa}{\the\day\ \shortmonthname[\the\month] \the\year}

\theoremstyle{plain}
\newtheorem{definition}{Definition}[section]
\newtheorem{theorem}{Theorem}[section]
\newtheorem{lemma}[theorem]{Lemma}
\newtheorem{conjecture}[theorem]{Conjecture}
\newtheorem{corollary}[theorem]{Corollary}
\newtheorem{exercise}{Exercise}[section]

\title{Fractal Tangency}
\author{poypoyan}
\date{\petsa\today}

\begin{document}

\maketitle

\begin{abstract}
Fractals are fascinating objects, but are still considered as pathological because most of the time, analytical techniques do not work on those. For example, the well-known ``inscribed square problem" is still unsolved for most fractal-like curves \cite{square}. In this short note we attempt to add something else to say to fractals by carefully distinguishing them from `smooth' curves. For now, this study is restricted to the continuum, but we provide some ideas for generalization to other sets.
\end{abstract}

\section{Intuition}
For a start, paths to a point are enough to determine fractality. For example, we can say that a surface is `fractal' if a path passing through a point inside that surface is fractal. When restricted to continuous curves in the continuum ($\mathbb{R}^n$), fractals can be thought of as exactly the nowhere differentiable functions. We somewhat follow that line of thought, but please remember a definition of the derivative: that it is the slope of the \textit{tangent line}. This means that differentiable curves will eventually look like a straight line when `zoomed in' to a point.

On the other hand, fractals are often viewed as `self-similar' (although it should not always be \cite{3b1b}!). Now observe that a straight line looks like itself when zoomed in. Thus, straight line is self-similar. In a sense, the straight line is to fractals as 0 is to positive natural numbers!

Since the derivative already captures \textit{tangency}, we can modify this to define \textit{fractal tangency}, and distinguish fractals from `smooth' curves. The following sections just develop a formalization to this idea.

\section{Definitions}
Let $C_n$ (for $n \ge 1$) be the set of all continuous functions $f:[0,1]\rightarrow\mathbb{R}^n$ such that $f(0)=(0)^n$ and $f(x) \ne (0)^n, \forall x \ne 0$, where $(0)^n$ is the origin on $\mathbb{R}^n$. Aside from the Cartesian coordinates, we make use of a variant of \textit{Hyper-spherical coordinates} (we shorten it as \textit{HSC} here). Let the set of HSCs be $S_n = \mathbb{R} \times [0, \pi]^{n-1}$ ($S_1 = \mathbb{R}$).

\begin{definition}\label{hsc}
Let $T_n: S_n \rightarrow \mathbb{R}^n$ (again, for $n \ge 1$) be the transformation from Hyper-spherical coordinates back to Cartesian coordinates such that:
\begin{itemize}
	\item $T_1(r) = r$.
	\item $T_{n+1}(r, \theta_1, \ldots, \theta_n) = ( \sin{\theta_n} \cdot T_n(r, \theta_1, \ldots, \theta_{n-1}), |r| \cos{\theta_n})$.
\end{itemize}
\end{definition}
\begin{exercise}[aka author is lazy]
Prove that $T_n$ is bijective \textit{except at origin} either by explicit construction of $T_n^{-1}$, or by showing surjection and injection. Please utilize induction if possible.
\end{exercise}
\begin{definition}
\textbf{Quasi-multiplication on HSC} is the binary operation $\cdot: S_n \times S_n \rightarrow S_n$ such that $$(r, \theta_1, \ldots, \theta_{n-1}) \cdot (s, \phi_1, \ldots, \phi_{n-1}) = (rs, \pi - |\pi - (\theta_1 + \phi_1)|, \ldots, \pi - |\pi - (\theta_{n-1} + \phi_{n-1})|).$$
\end{definition}
\begin{definition}
\textbf{Quasi-division on HSC} is the binary operation $/: S_n \times S_n \rightarrow S_n$ such that $$(r, \theta_1, \ldots, \theta_{n-1}) / (s, \phi_1, \ldots, \phi_{n-1}) = (r/s, |\theta_1 - \phi_1|, \ldots, |\theta_{n-1} - \phi_{n-1}|).$$
\end{definition}
Notice the operations on angles. The idea is that if the sum/difference of angles is outside of $[0,\pi]$, we put it back to $[0,\pi]$ by `remeasuring' it from angle $0$.

\section{Equivalence Relation}
Now we can define the main concept to distinguish the rough from the smooth:
\begin{definition}\label{equiv}
Two continuous paths to the origin $f, g \in C_n$ have the `same tangency at origin' (we notate as $f \sim_n g$) if there exists a `hyper-rotation' $r = (1, \theta_1, \ldots, \theta_{n-1}) \in S_n$ such that $$\lim_{x\rightarrow 0^{+}} \frac{r \cdot T_n^{-1}(f(x))}{T_n^{-1}(g(x))} = (1, (0)^{n-1}).$$
\end{definition}
\begin{theorem}
$\sim_n$ is an equivalence relation on $C_n$.
\end{theorem}
\begin{proof}
For reflexivity, the $r$ for $f \sim_n f$ would be $(1, (0)^{n-1})$. For symmetry, given the $r$ for $f \sim_n g$, the $r'$ for $g \sim_n f$ would be $(1, (0)^{n-1})/r$. Lastly for transitivity, given the $r$ for $f \sim_n g$ and $r'$ for $g \sim_n h$, the $r''$ for $f \sim_n h$ would be $r \cdot r'$.
\end{proof}
\begin{corollary}
$\sim_n$ partitions $C_n$ into the set of equivalence classes $C_n / \sim_n$.
\end{corollary}
One such class is the class $L_n$ of `smooth' paths to a point, or more informatively, the class of paths with the \textit{tangent line} at origin.
\begin{definition}
For a continuous path to the origin $f \in C_n$, $f \in L_n$ iff $f \sim_n (x, (0)^{n-1})$.
\end{definition}
Now we can finally distinguish fractals, and define the title of this note:
\begin{definition}
A continuous path to the origin $f \in C_n$  has \textbf{fractal tangency} at origin iff $f \not\in L_n$.
\end{definition}

\section{Other Ideas}
Here are some other ideas/comments for further study of the concepts outlined in previous sections:
\begin{itemize}
	\item The limit in Definition \ref{equiv} can be simplified to $$\lim_{x \rightarrow 0^+}\frac{T_n^{-1}(f(x))}{T_n^{-1}(g(x))} = (1, \theta_1, \ldots, \theta_{n-1}).$$ This suggests a weakening of the equivalence relation: the limit should now be $(r, \theta_1, \ldots, \theta_{n-1})$ for any $r \ne 0$. However, at the time of writing, we do not have a good `geometric' interpretation for this.
	\item Of course there are fractals beyond $C_n$ like the well-known Sierpiński carpet! The first idea is that paths on \textit{sets} can be defined as $f:[0,1] \rightarrow 2$. Fortunately there are no notions of continuity, HSC, and hyper-rotation anymore. Then the next idea is to modify the $\epsilon$ part of the $\epsilon$-$\delta$ definition: instead of $|f(x)-L|<\epsilon$, we simply have $f(x) \oplus L = 0$, where $\oplus$ is the XOR operation. Now the only perfectly `smooth' paths are $f_0(x)=0$ and $f_1(x)=1$, for all $x\in[0,1]$. The last idea is to define a suitable (quasi-)multiplication and (quasi-)division: maybe the XOR operation can also be used for that. Hopefully we can now define the analogue for Definition \ref{equiv}. These ideas should generalize to other sets aside from $\mathbb{R}^n$ and $2$.
\end{itemize}

\bibliographystyle{plain}
\bibliography{fractal}
\end{document}